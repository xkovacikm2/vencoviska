\documentclass[12pt,oneside,slovak,a4paper]{article}
\usepackage[slovak]{babel}
\usepackage[utf8]{inputenc}
\usepackage[T1]{fontenc}
\usepackage{graphicx}
\usepackage{listings}
\usepackage{url} % príkaz \url na formátovanie URL
\usepackage{hyperref} % odkazy v texte budú aktívne (pri niektorých triedach dokumentov spôsobuje posun textu)
\usepackage{longtable} 
\usepackage{cite}
\usepackage{times}
\usepackage{listings} % umožňuje paste kódu \lstset{language=Java} \begin{lstlisting}[frame=single]
\usepackage[dvips,dvipdfm,a4paper,centering,textwidth=14cm,top=4.6cm,headsep=.6cm,footnotesep=1cm,footskip=0.6cm,bottom=3.8cm]{geometry}
%opening
\pagestyle{headings}
\title{DBS etapa 1}

\author{Michal Kováčik\\[2pt]
	{\small Slovenská technická univerzita v Bratislave}\\
	{\small Fakulta informatiky a informačných technológií}\\
	{\small \texttt{xkovacikm2@gmail.com}}
	}

\date{\small datum}

\begin{document}

\maketitle

\section{Stručný prehľad}
Aplikácia je vo veľmi rozpracovanom štádiu. Je možné sa do nej registrovať, prihlásiť, prezerať zoznamy miest, prezerať oblasti v mestách, prezerať, pridávať a mazať vlastné komentáre v jednotlivých zónach, prezerať profily iných užívateľov, a pokiaľ je človek administrátor, môže aj mazať iných užívateľov.
Interaktívne zaznačovanie zón na mape zatiaľ nie je implementované, ale to by pre účely tohoto odovzdania nemalo prekážať.
Celý projekt je implementovaný v ruby a využíva postgresql databázu.

\section{Realizácia scenárov}
\paragraph{Vytvorenie nového záznamu} je realizované ako registrácia uživateľa do systému a pridanie komentáru (využíva vlastný insert my\_save) k ľubovoľnej zóne.
\paragraph*{Aktualizácia existujúceho záznamu} je realizovaná ako možnosť upraviť uživateľské informácie ako meno, email, popis atď... (využíva vlastnú implementáciu my\_update\_attributes)
\paragraph{Vymazanie záznamu} je realizovaný ako zmazanie uživateľa administrátorom (my\_destroy, ktorá zároveň zmaže všetky uživateľove komentáre) a ako zmazanie vlastného komentára uživateľom.
\paragraph{Zobrazenie prehľadu viacerých záznamov} je realizované ako prehľad užívateľov, prehľad miest, prehľad zón v mestách a prehľad komentárov v zónach.
\paragraph{Zobrazenie konkrétneho záznamu} je realizované ako zobrazenie konkrétneho uživateľa, konkrétneho mesta, alebo konkrétnej zóny.
\paragraph{Transakcia} implementovaná ako vymazanie všetkých uživateľových komentárov a následne zmazanie záznamu samotného uživateľa.
\paragraph{Join, agregačná funkcia, Group by} sú použité pre zobrazenie prehľadu počtu registrovaných uživateľov v meste a priemerného veku uživateľov prihlásených v danom meste.
\paragraph{Filtrovanie} realizované implicitne ako zobrazovanie oblastí len pre dané mesto, komentárov len pre danú oblasť, alebo uživateľa.

\section{Model}
\includegraphics[scale=0.7]{VOS_DBS-model.png}
Model som oproti pôvodnému návrhu, po konzultácií, zjednošil. Vypadli odtiaľ zbytočné väzobné entity medzi userom a komentárom, userom a mestom a tabuľka rolí pre uživateľa. User vytvára komentár, môže ich mať ľubovoľne veľa. User vytvára aj Area, takisto ľubovoľne veľa. Area vždy patrí do nejakého mesta a má ohraničenie niekoľkými rovnými čiarami.

\section{Záver}
Keďže aplikácia má nad 5MB limit AIS, prikladám link na stiahnutie z google drive \url{https://drive.google.com/open?id=0B7NOCTI1dZlfcUdGMjEzbS1XMWM}
K projektu som priložil dump aj seed do databázy.

\end{document}